\chapter{Output and Input}\label{CHAP_OutAndIn}

\section{Difficulty: EASY}


\subsubsection*{Exercise 2.E01}

Determine the result of:
\begin{enumerate}[label=(\alph*)]
	\item $245 + 129$
	\item $501 - 333$
	\item $37 \cdot 11$
	\item $99 / 33$
	\item $15 + 777 \cdot 2$
	\item $30 / (4 + 1)$
\end{enumerate}


\textit{Hints:
Use addition, subtraction, multiplication, and division operators. Remember that brackets
can change the order in which mathematical expressions are evaluated.}\\[1cm]


% ------------------------------------------------------------------------------


\subsubsection*{Exercise 2.E02}
Determine the primitive data type of the following expressions:
\begin{enumerate}[label=(\alph*)]
	\item {\code{100}}
	\item {\code{0}}
	\item {\code{“Python”}}
	\item {\code{4.0}}
	\item {\code{6 / 2}}
	\item {\code{“2017”}}
\end{enumerate}


\textit{Hints:
You can try to determine the primitive data type without using Python first. If you get stuck, remember the {\code{type()}} function.}\\[1cm]


% ------------------------------------------------------------------------------


\subsubsection*{Exercise 2.E03}
Write a short program that produces the following output:
\begin{center}
	{\code{Hello World!}}
\end{center}


\textit{Hints:
Use the {\code{print()}} function.
}\\[1cm]

% ------------------------------------------------------------------------------


\subsubsection*{Exercise 2.E04}

Write a short program that prints your name, your favourite food and a hobby. For example:
\begin{center}
	{\code{Hello. My name is Cookie Monster. I like cookies and my\\ favourite hobby is not sharing my cookies.}}
\end{center}


\textit{Hints:
Use the {\code{print()}} function.}\\[1cm]


% ------------------------------------------------------------------------------


\subsubsection*{Exercise 2.E05}
Write a short program that produces the following output:
\begin{center}
	{\code{Hello World!}}
\end{center}
But this time, use a \textbf{variable}.\\


\textit{Hints:
To generate a variable you will need the assignment operator {\code{=}}. Once you have stored the string inside the variable you can use the {\code{print()}} function to display it.}\\[1cm]


% ------------------------------------------------------------------------------


\subsubsection*{Exercise 2.E06}
Define two variables $a = 11$ and $b = 599$. Determine:
\begin{enumerate}[label=(\alph*)]
	\item  $a + b$
	\item  $a - b$
	\item  $a \cdot 10$
	\item  $b / 4$
	\item  $a + 15 - b \cdot 3$
\end{enumerate}


\textit{Hints:
You will need to initialize two variables and then apply the addition, subtraction, multiplication and division operator to them.}\\[1cm]


% ------------------------------------------------------------------------------


\subsubsection*{Exercise 2.E07:}
Decide whether the following variable declarations are valid or not:

\begin{enumerate}[label=(\alph*)]
	\item {\code{\_ingredient = “Milk, flour, eggs”}}
	\item {\code{london-population = 8800000}}
	\item {\code{n = 3.14159}}
	\item {\code{string = 100}}
	\item {\code{apples\_amount = 10}}
	\item {\code{uk\_\_\_\_\_\_\_\_\_capital = “London”}}
	\item {\code{123456 = “123456”}}
\end{enumerate}


\textit{Hint:
If you are uncertain, you could always try and type the variable declaration into a Jupyter Notebook. Try and run the command and check whether Python returns an error.}\\[1cm]


% ------------------------------------------------------------------------------


\subsubsection*{Exercise 2.E08}
Write a short program that asks the user for their name and then simply prints it to the screen.\\


\textit{Hints:
You will need the {\code{input()}} and the {\code{print()}} function. You will also need to use a variable.}\\[1cm]


% ------------------------------------------------------------------------------

\newpage
\section{Difficulty: MEDIUM}


\subsubsection*{Exercise 2.M01}
Write a code that asks the user for their name and where they were born (city and country).
Greet them with a message of the following format:\\
\begin{center}
	\code{“Hello [NAME] from [CITY] in [COUNTRY]! Nice to meet you.”}
\end{center}


\textit{Hints:
Use the {\code{input()}} function to request user input and the {\code{print()}} function to display output on the screen. Remember that you can combine strings using the {\code{+}} operator (even within the {\code{print()}}) function.}\\[1cm]


% ------------------------------------------------------------------------------


\subsubsection*{Exercise 2.M02}
Create a python script and using one command line only reproduce the following output:\\


\hspace*{5mm}{\code{Night is now falling}}\\
\hspace*{10mm}{\code{So ends this day}}\\
\hspace*{15mm}{\code{The road is now calling}}\\
\hspace*{20mm}{\code{And I must away}}\\


\textit{Hint:
You can use the {\code{print()}} function and spaces to reproduce above output. However, there is a quicker way of doing it: You can use the string formatters {\code{\textbackslash n}} and {\code{\textbackslash t}} to generate new lines and tabs.}\\[1cm]


% ------------------------------------------------------------------------------


\subsubsection*{Exercise 2.M03:}
In this exercise you will generate a shopping list. You already know what you want to buy:\\
Apples, tomatoes, cucumbers, beans, and eggs. But you will need to ask the user how many apples, tomatoes, cucumber etc. they want.\\
Once you know the required number of each item print a shopping list that looks something like this:\\


{\code{Shopping list:}}\\
\hspace*{5mm}{\code{- Apples: 10}}\\
\hspace*{5mm}{\code{- Tomatoes: 5}}\\
\hspace*{5mm}{\code{- Cucumbers: 1}}\\
\hspace*{5mm}{\code{- Can of beans: 1}}\\
\hspace*{5mm}{\code{- Eggs: 6}}\\


\textit{Hints:
You will need the {\code{input()}} and the {\code{print()}} function several times. Alternatively, you can use the string formatters {\code{\textbackslash t}} and {\code{\textbackslash n}}. Remember that the \code{+} operator concatenates strings.}\\[1cm]


% ------------------------------------------------------------------------------


\subsubsection*{Exercise 2.M04}
Copy the following commands into Jupyter Notebook. Run them one by one and try to
understand the error message then fix them:\\

\hspace*{5mm}(a) {\code{print(“Goodbye World’)}}\\
\hspace*{5mm}(b) {\code{\# Calculating 1 + 99:}}\\
\hspace*{12mm}{\code{print(99 + 1 =)}}\\
\hspace*{5mm}(c) {\code{\# We can also print several strings within}}\\
\hspace*{12mm}{\code{one print command:}}\\
\hspace*{12mm}{\code{print(“Hello World” + “Goodbye World”)}}\\


\textit{Hints:
Python will usually show you rather accurately where the mistake is (either by giving the line or with a little arrow). Make sure to use this!}\\[1cm]


% ------------------------------------------------------------------------------


\subsubsection*{Exercise 2.M05 \red{[M]}}
Determine the result of:
\begin{enumerate}[label=(\alph*)]
	\item $(5+45)-(70-4+11)$
	\item $10 \cdot 4^3$
	\item $\frac{15}{0.1} + \frac{0.001}{34+22}$
	\item $\frac{4}{1/5 +16} / \frac{10+3}{8\cdot 11 - 13/27}$
\end{enumerate}


\textit{Hints:
Rewrite the mathematical symbols into addition, subtraction, multiplication, division operators. Remember to use brackets were appropriate.}\\[1cm]


% ------------------------------------------------------------------------------


\subsubsection*{Exercise 2.M06}
Write a code that asks the user for their name and their age. Print a message greeting them and telling them in which year they will turn 100.\\


\textit{Hints:
Use the {\code{input()}} function to request user input and the {\code{print()}} function to display output on the screen. Remember that the {\code{input()}} function always returns a string. If you want to perform a mathematical operation on an input you will need to convert it into a number first using either {\code{int()}} or {\code{float()}.}}\\[1cm]


% ------------------------------------------------------------------------------


\subsubsection*{Exercise 2.S01}
Kermit, the Cookie Monster, and Big Bird have found a cookie stash. Kermit takes 10 cookies,
Big Bird 15 and the Cookie Monster 45. Using variables, calculate how many cookies there are in total and display the result.\\


\textit{Hints:
You will need a variable for the number of cookies taken by each character and another
variable for the total amount of cookies. Use the {\code{print()}} function to display the result but remember type conversion of a number into a string via {\code{str()}} if necessary.}\\[1cm]


% ------------------------------------------------------------------------------


\subsubsection*{Exercise 2.S02}
Before a company can send out a delivery to a costumer, the name and address of the
costumer need to be determined. Write a code that asks the costumer for their name,
address, postcode, city and country. Then display the address on the screen.\\


\textit{Hints:
User the {\code{input()}} function to obtain user input and the {\code{print()}} function to display information.}\\[1cm]

% ------------------------------------------------------------------------------



\newpage
\section{Difficulty: HARD}

\subsubsection*{Exercise 2.H01 \red{[M]}}
Write a code that requests two numbers from the user. Store the first number in variable
{\code{var1}} and the second number in {\code{var2}}.
Swap the values of the variables so that {\code{var1}} contains the second number and {\code{var2}} the
first
\begin{enumerate}[label=(\alph*)]
	\item using an additional variable {\code{var3}}
	\item without using an additional variable
\end{enumerate}
Can you think of a reason why (a) is better than (b)?\\


\textit{Hints:
Use the {\code{input()}} function to request user input. Remember to convert the strings returned by the {\code{input()}} function into numbers using either {\code{int()}} or {\code{float()}} if you want to perform mathematically operations on them. Try playing around with additions and subtractions to exchange the variable content without having to use an additional third variable.}\\[1cm]


