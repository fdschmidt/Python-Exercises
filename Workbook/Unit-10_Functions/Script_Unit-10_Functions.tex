\chapter{Bonus: Functions}\label{CHAP_Functions}

\section{Difficulty: MEDIUM}


\subsubsection*{Exercise 10.01}

Generate a function {\code{PrintMessage()}} that prints a message of your choosing to the screen
when called.\\

\textit{Hints:
Remember to include the def and the : when initializing a function.}\\[1cm]


% ------------------------------------------------------------------------------

\subsubsection*{Exercise 10.02}
Generate a function {\code{PrintUserInput(userInput)}} that accepts one argument (for
example a user input) and prints it to the screen.\\

\textit{Hints:
Use the {\code{input()}} function to request user input.}\\[1cm]


% ------------------------------------------------------------------------------

\subsubsection*{Exercise 10.03}
Generate a function {\code{CalculateSum(a, b, c, d)}} that takes four arguments {\code{a}}, {\code{b}}, {\code{c}}, and {\code{d}} (either integers or floats) and returns the sum {\code{a + b + c + d}}.\\

\textit{Hints:
Don’t forget the return keyword if you want your function to return a value to the code.}\\[1cm]


% ------------------------------------------------------------------------------

\subsubsection*{Exercise 10.04}
Generate a function {\code{CalculateAge(birthyear)}} that takes a persons birth year and
returns their current age.\\

\textit{Hints:
If you want to include user input, remember to convert strings to integers before
performing any mathematical operations on it.}\\[1cm]


% ------------------------------------------------------------------------------

\subsubsection*{Exercise 10.05}
Generate a function {\code{FindMaximumInList(list)}} that accepts a list of any length and
returns the maximum number in the list.\\
Bonus:\\
Generate a function {\code{FindMinimumInList(list)}} that accepts a list of any length and
returns the minimum number in the list.\\

\textit{Hints:
Remember you can iterate over a list with either a for or a while loop.}

